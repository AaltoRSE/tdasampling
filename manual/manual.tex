\documentclass[11pt]{article}
\usepackage{url}
\usepackage{amsmath}
\usepackage{array}
\usepackage{amsfonts}
\usepackage{enumerate}
\usepackage{graphicx}
\usepackage{wrapfig} 
\usepackage{xcolor} 
\usepackage{stmaryrd}
\usepackage{tipa}
\usepackage{float}
\usepackage{bbm} 
\usepackage{tikz}
\usepackage{algpseudocode}
\usepackage{algorithm}
\usepackage{mathtools}
\usepackage{afterpage}
\usepackage{pdflscape}
\usepackage{capt-of}
\usepackage{epstopdf}
\usepackage{setspace}
\usepackage{changepage}
\usepackage{tabularx}
\usepackage{geometry}
\usepackage{fancyhdr}
\usepackage{amssymb}
\usepackage{tikz-cd}
% Let top 85% of a page contain a figure
\renewcommand{\topfraction}{0.85}

% Default amount of minimum text on page (Set to 10%)
\renewcommand{\textfraction}{0.1}

% Only place figures by themselves if they take up more than 75% of the page
\renewcommand{\floatpagefraction}{0.75}

% Basic math operators that are probably the most universal to 
% all documents
\newcommand{\mz}{\mathbb{Z}}
\newcommand{\mr}{\mathbb{R}}
\newcommand{\mn}{\mathbb{N}}
\newcommand{\mq}{\mathbb{Q}}
\newcommand{\mc}{\mathbb{C}}
\newcommand{\del}{\partial}
\newcommand{\twid}{{\raise.17ex\hbox{$\scriptstyle\sim$}}}
\DeclareMathOperator{\im}{Im}
\DeclareMathOperator{\ext}{Ext}
\DeclareMathOperator{\Hom}{Hom}
\DeclareMathOperator{\coker}{coker}
\newcommand{\id}{\mathbbm{1}}
\newcommand{\si}{\mathcal{I}}

% Homological algebra 
\newcommand{\ilim}{\varprojlim}
\newcommand{\limone}{\varprojlim^1\nolimits}
\newcommand{\dlim}{\varinjlim}
\newcommand{\mzp}{\mz_{p^\infty}}  % The limit of Z/p -> Z/p^2 -> ...

% Covering spaces
\newcommand{\xtil}{\tilde{X}}
\newcommand{\lxtil}{\tilde{x}}
\newcommand{\inv}{^{-1}}
\newcommand{\btil}{\tilde{b}}
\newcommand{\ltil}{\tilde{\ell}}

% Category theory 
\newcommand{\mcalc}{\mathcal{C}}
\newcommand{\cop}{\mathcal{C}^{\text{op}}}
\newcommand{\bset}{\textbf{Set }}
\newcommand{\bpos}{\textbf{Pos }}
\newcommand{\crel}{\textbf{CRel }}
\newcommand{\lbrac}{\llbracket}
\newcommand{\rbrac}{\rrbracket}
\newcommand{\pfin}{\mathbf{P_{fin}}}
\newcommand{\upv}{\Lambda}
\newcommand{\app}{\text{App}}
\newcommand{\alar}[1]{\xrightarrow{\alpha_{#1}}}
\newcommand{\arr}[1]{\xrightarrow{#1}}
\newcommand{\al}[1]{\xleftarrow{#1}}
\newcommand{\all}[1]{\xleftarrow{\alpha_{#1}}}


% Machine Learning
\newcommand{\xbold}{\mathbf{x}}
\newcommand{\xibold}{\mathbf{x}_i}
\newcommand{\ybold}{\mathbf{y}}
\newcommand{\mubold}{\mathbf{\mu}}
\newcommand{\xtrans}{\xbold^{\mathbf{T}}}
\newcommand{\wbold}{\mathbf{w}}
\newcommand{\wsub}{w_0}
\newcommand{\iprob}{p^{a_i}_{b_i,c_i}}
\newcommand{\simb}{M_{a_i,b_i}}
\newcommand{\simc}{M_{a_i,c_i}}
\DeclareMathOperator*{\argmin}{argmin}
\DeclareMathOperator{\sgn}{sgn}
\DeclareMathOperator{\softmax}{softmax}


% TDA 
\DeclareMathOperator{\pix}{pix}
\DeclareMathOperator{\mult}{mult}
\DeclareMathOperator{\topo}{\textbf{Top}}
\DeclareMathOperator{\pair}{\textbf{Pair}}
\DeclareMathOperator{\fdv}{\textbf{FDVec}}

% Algebraic geometry
\DeclareMathOperator{\mv}{\mathcal{V}}
\DeclareMathOperator{\mvr}{V_\mr}



% Thesis
\DeclareMathOperator{\minp}{\textsc{MinPoint}}
\DeclareMathOperator{\critp}{\textsc{CritPoint}}
\DeclarePairedDelimiter{\ceil}{\lceil}{\rceil}
\DeclarePairedDelimiter{\floor}{\lfloor}{\rfloor}

\usepackage{amsmath}
\usepackage{amsthm}
\usepackage{amssymb,amsfonts,amscd}
\usepackage{graphicx}
\usepackage{minted} 

%=================================================


\textwidth 6.5in
\textheight 8.75in
\oddsidemargin 0in
%\evensidemargin 0in
\topmargin -0.5in

%\usepackage[notref,not]{showkeys}

%=================================================





%=================================================
% new commands, math operators
%=================================================

\newcommand{\sA}{{\mathcal A}}
\newcommand{\sB}{{\mathcal B}}
\newcommand{\sC}{{\mathcal C}}
\newcommand{\sE}{{\mathcal E}}
\newcommand{\sI}{{\mathcal I}}
\newcommand{\sJ}{{\mathcal J}}
\newcommand{\sL}{{\mathcal L}}
\newcommand{\Var}{{\mathcal V}}
\newcommand{\sZ}{{\mathcal Z}}
\newcommand{\sU}{{\mathcal U}}
\newcommand{\sW}{{\mathcal W}}

\newcommand\pn[1]{{\bP}^{#1}}
\newcommand\red[1]{X^{[#1]}_{(1,\dots, 1)}}

\newcommand\bC{{\mathbb C}}
\newcommand\bN{{\mathbb N}}
\newcommand\bP{{\mathbb P}}
\newcommand\bQ{{\mathbb Q}}
\newcommand\bR{{\mathbb R}}
\newcommand\bZ{{\mathbb Z}}

\newcommand\suchthat{~|~}

\DeclareMathOperator*{\argmax}{arg\,max}
\DeclareMathOperator*{\trace}{\rm trace}
\DeclareMathOperator*{\rank}{\rm rank}



%===============================================
%          Title
%===============================================


\begin{document}
\title{\texttt{tdasampling} User's Manual}

\author{
Parker Edwards
}

\maketitle
\tableofcontents

This manual is for Linux distributions, which is the only OS for which \texttt{tdasampling} has been tested.

\section{Installation} 
Installation is via pip. In this example, the current working directory is a 
virtual environment. It will install the scripts to the directory's bin folder rather than into the global bin folder.

\begin{minted}{shell} 
$ pip install tdasampling
\end{minted} 

The pip command will install the Python dependencies automatically.

\section{Installing requirements}
\begin{minted}{shell} 
$ /bin/bash
\end{minted} 
\subsection{\texttt{libspatialindex} and \texttt{rtree}} 

\subsubsection{Libspatialindex}
If you have root access to the machine you're using, you can follow the standard installation instructions. First you need to install libspatialindex, with installation instructions at \url{https://libspatialindex.github.io/install.html}. 

So, for instance, having changed your working directory to wherever you want to place the source files before installation: 
\begin{minted}{shell} 
$ curl http://download.osgeo.org/libspatialindex/spatialindex-src-1.8.5.tar.gz | tar xvz
$ cd spatialindex-src-1.8.5/
$ cmake .
$ make 
$ sudo make install
\end{minted} 

If you don't have root access to the installation machine, you'll want to install libspatialindex into a non-standard location. You do not need to place the \texttt{libspatial} source files in the same directory.

\begin{minted}{shell} 
$ curl http://download.osgeo.org/libspatialindex/spatialindex-src-1.8.5.tar.gz | tar xvz
$ cd spatialindex-src-1.8.5/
$ cmake -DCMAKE_INSTALL_PREFIX=/home/parker/test_dir/local .
$ make 
$ make install
\end{minted} 

\subsubsection{Adjusting rtree if libspatialindex is installed without root access}
The \texttt{Rtree} Python package provides a Python interface over \texttt{libspatialindex} using Ctypes. If you have installed the \texttt{libspatialindex} library non-globally (without root access), then it has a hard time locating the library files. The \texttt{Rtree} package provides an environment variable you can export in your shell to fix this, but you have to export it every time you want to use \texttt{tdasampling}. An example: 

\begin{minted}{shell} 
$ export SPATIALINDEX_C_LIBRARY="/home/parker/test_dir/local/lib/
libspatialindex_c.so.4"
\end{minted} 

As an alternative, you can just hard code the non-standard location of your libspatialindex files via the following command. If you change the location of libspatial, you'll need to update it again. The general form is: 

\begin{minted}{shell} 
$ sed -i "s/find_library('libspatialindex_c')/'\/a\/path\/to\/
libspatialindex_c\.so\.4'/g" /a/path/to/your/rtree/source/core.py
\end{minted} 

The escape characters are unfortunately necessary from the command line. A specific example is:

\begin{minted}{shell} 
$ sed -i "s/find_library('spatialindex_c')/'\/home\/parker\/test_dir\/
local\/lib\/libspatialindex_c.so.4'/g" \
/home/parker/test_dir/lib/python2.7/site-packages/rtree/core.py
\end{minted} 

If you make this adjustment, you won't have to worry about exporting environmental variables every time you want to use \texttt{rtree}.

\subsection{MPI} 
Follow the installation instructions for MPI at \url{http://www.mpich.org/downloads/} or another favorite MPI implementation. 

\subsection{Bertini} 
Bertini offers both binary and source downloads at \url{https://bertini.nd.edu/download.html}. If you were using a 64-bit Linux machine, you could download and unpack the binaries using: 

\begin{minted}{shell} 
$ curl https://bertini.nd.edu/BertiniLinux64_v1.5.1.tar.gz | tar xvz
\end{minted} 

Optionally, you can also add the binary folder to your PATH by modifying \textasciitilde /.bashrc (replace \textasciitilde /.bashrc with the appropriate file for your system). You can do the same for mpiexec. 


\section{Quick start tutorial} 
The quick start will be sampling a quartic variety in 3 variables and 1 equation:

\[
4x^4+7y^4+3z^4-3-8x^3+2x^2y-4x^2-8xy^2-5xy+8x-6y^3+8y^2+4y = 0
\]




\section{Advanced settings}

\subsection{Parallelism}









\end{document}

